\section{Historical Background and Motivation}
\label{sec:1}
Since the dawn of humankind, when our ancestors first started to investigate the secrets of Mother Nature, one particular topic of interest for them was the mystery of flight. They yearn to fly like birds, but more often than not, their attempts were met with unsatisfactory results. Even our modern term of aviation derived from the word \textit{avis}, which is Latin for bird. From the 5\textsuperscript{th} century kite of Mo Tze, an Ancient Chinese philosopher, to the highly agile fighter planes dueling over the skies of Europe in the earlier half of 20\textsuperscript{th} century, humans have always tried to imitate the flight of birds, insects, and other flying beasts.\par
While it may not be obvious then, biological aviators does have favorable characteristics compared to man-made machines. One such characteristics is that biological flyers tend to have high manouverability. Some birds are able to hold 14G of forces, which is spectacular considering that modern fighter aircrafts can only withstand 10G of forces. \citet{weisfogh} quantitatively studied biological flight by investigating lift generation mechanism of \textit{Encarsia formosa}, which was later called as clap-and-fling mechanism. \citet{ellington} also put forward vortex model where insect wings are modeled as propeller blades in quasi-steady condition. Ellington observed that there exist a lift enhancement mechanism called delayed stall caused by the formation of leading edge vortex, which was later confirmed by \citet{dickinson}. Another kinematics of flapping, rotational forces and wake capture, was also identified by \citet{sane} as a lift enhancement parameter. Rotational force increase lift if the wing rotates at the end of each half-strolke and wake capture increase lift when the vortex of an initial stroke hits vortex shed by stroke reversal motion.\par
Aquatic creatures have also shown this remarkable characteristics underwater not unlike their flying cousins. \citet{hu}, \citet{tan}, and \citet{yu} has proved that flapping motion of fish fin utilised for generating thrust on an unmanned underwater vehicle leave considerably less wake compared to traditional propeller-based propulsion system, making it less susceptible to detection. Their manouverability and power consumption shows favorable improvement compared to previous design. The application of natural flyers and swimmers in man-made machines is a Herculian task, since their wing and fin move in complex ways that is difficult to imitate. Their bodies flap, twist, flex, sweep, and plunge in order to move their body and to hold their own weight, thus generating lift and propulsive force simultaneously. This motion produce complex vortex formation and interaction, which is a challenging aerodynamic topic on it's own.\par
Optimisation of caudal fin concerns about maximising thrust and propulsive efficiency by varying design variable. Since the number of design variable in flapping fin is massive, finding a mathematical function that relates thrust force to said design variable is difficult, which renders conventional gradient-based optimisation unsuitable. The proposed optimisation method is Genetic Algorithm (GA), a global non-gradient based method that is based on the  principle of genetic, evolution and natural selection.\par
In Aerogasdynamics Laboratory of Bandung Institute of Technology, GA has been developed and tested on various engineering cases. \citet{adhynugraha} and \citet{dwianto2} used GA to optimise airfoil geometry for design purpose. \citet{nguyen}, \citet{nguyen2} and \citet{wuwung} showed that flow characteristics such as velocity and vorticity field are heavily influenced by flapping plate kinematics. It was also demonstrated that certain kinematics can enhance force generated by the flapping plate \citep{nguyen}. \citet{palar} then developed Micro-GA which was coupled with experimental method to optimise flapping flight kinematics for maximum thrust generation. It was also suggested that optimisation alone is not enough, parametric study is needed to study the effect of flapping kinematics on force production adequately \citep{palar2}.\par
Experimental method used to optimise biomimetics propulsion system in ITB began with Daniella's master thesis \citep{dea}. She optimised the lift of a flapping 2D plate by studying the effect of the plate's kinematics to the generation of lift. The optimisation method used by Daniella was surrogate assisted genetic algorithm and the fitness test was done by using towing tank experiment. Her finding is that certain kinematic parameters, rotational timing and angle of attack heavily influence the generation and consequently, the improvement of lift. She also finds out through PIV analysis that rotation prior to half-stroke greatly enhance the lift generated by the plate. Certain range of angle of attack also maximise lift due to delayed stall, therefore sustaining the lift. \citet{luqman} applied G-SAGA with kriging optimisation method to obtain optimum time-averaged net thrust force with fin flexibility as the design consideration. It is found that optimum frequency decrease as fin panel's flexibility decrease, with a particular value of flexibility produce the highest optimum value of time-averaged net thrust force. Through PIV experiments it is also observed that increase of amplitude lateral displacement of the fin base will produce higher time-averaged net thrust force. The resulting flow patterns and vorticity contour reveals that in order to obtain optimum thrust, phase difference between the leading edge and the trailing edge of the oscillating fin should be 90 degrees.\par
Research done by \citet{Esposito56} shows that there exist an optimum stiffness configuration corresponding to a certain flapping frequency that produce higher mean thrust value. The same argument was made by \citet{luqman} who argues that for a given fin panel with a stiffness value, there exist a corresponding optimum frequency value which decrease as the stiffness decrease. The optimum thrust, however increase up to a certain point until the thrust also decrease. This research focus only on the optimum frequency of a given fin panel; granted the stiffness of fin panel models used in this thesis was varied, but as it were, stiffness optimisation takes a back seat to frequency optimisation. This thesis seeks to provide an insight on which stiffness configuration of a fin panel which yields the most optimum thrust and also on why such configuration produce highest thrust. To find the value of the optimum thrust, G-SAGA optimisation strategy was used and to explain the result PIV flow visualisation technique was utilised. Ultimately, this thesis seeks to contribute to flapping fin design for UUV application from an aerodynamicist point of view and its correlation to the mechanism of a flapping body.
\section{Research Objective}
\label{sec:2}
The objectives of this thesis are as follows:
\begin{enumerate}
    \item To implement Global Surrogate-Assisted Genetic Algorithm (G-SAGA) to optimise thrust generated by a fin panel by varying the fin panel's stiffness and kinematical parameters.
    \item To find the correlation between fin stiffness and kinematical parameters.
\end{enumerate}
\section{Scope of Work}
\label{sec:3}
To limit this thesis, the scope of work on this thesis is outlined as follows:
\begin{enumerate}
    \item Observed kinematical parameters were velocity and frequency.
    \item Velocity was varied from 0.15 cm/s to 4 cm/s and frequenct was varied from 0.5 Hz to 2 Hz.
    \item Evaluation of fitness value was conducted using towing tank experiments.
    \item Fin movement was produced by translational and rotational mechanism using Arduino.
    \item Optimisation was conducted using G-SAGA genetic algorithm as the optimisation algorithm and kriging as the surrogate modeling method in Python.
    \item Data acquisition was done in MATLAB R2015a by using NI DAQmx libraries.
\end{enumerate}
\section{Research Methodology}
\label{sec:4}
There are three major parts to this thesis: experimental system development and optimisation. The experimental system itself is divided further into three major subsystems: translational, rotational, and force measurement system. The fin panel was translated using stepper motor and belt-and-pulley system and rotated using a servo motor to reproduce fish fin oscillation motion in a towing tank filled with water. The force experienced by the plate was measured using a load cell. A code in Arduino and MATLAB R2015a was made to control the traverse of the observed specimen via Arduino UNO board and to record force data from load cell via National Instruments data acquisition device according to kinematics parameter input, respectively. After the experimental system to evaluate fitness (in this case, force) measurement had been ready, optimisation was done using single objective G-SAGA with Kriging developed by \citet{kadal}.
\section{Thesis Outline}
\label{sec:5}
This thesis contains 5 chapters. The general outline and brief description of each chapters are outlined as follows:
\begin{itemize}
    \item Chapter I: Introduction

    This chapter contains historical background, research objectives, scope of work, methodology, and the outline of the thesis.
    \item Chapter II: Fundamental Theory

    This chapter contains explanations on the theoretical background of oscillating fin and genetic algorithm (GA) and surrogate modeling technique, particularly Kriging method. Useful terminology, Navier-Stokes equation of unsteady oscillating kinematics, wake structure, scaling parameters used in oscillating fin will be discussed. The discusson continues with coupling of GA and global surrogate modeling technique that was used in this research.
    \item Chapter III: Experimental System Design and Setup

    This chapter contains explanation about the design process and setup of experimental system which consists of translational and rotational motion control, force measurement, and image acquisition system. Design choice, design problems and solution also flow visualisation method used in the present thesis is also explained.
    \item Chapter IV: Result and Analysis

    This chapter begins with thrust measurement system for fitness calculation during optimisation process. The couple of G-SAGA with experimental setup, elaboration on each test case and it's optimum solution along with flow physics obtained from PIV experiments, and discussion about the results close this chapter.
    \item Chapter V: Conclusion and Future Works

    This chapter concludes present research and recommendation regarding future research.
\end{itemize}