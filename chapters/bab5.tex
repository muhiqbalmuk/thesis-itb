\section{Conclusion}
G-SAGA with kriging has been implemented succesfully to predict the optimum flexibility distribution across the fin's spanwise direction, with error never exceeding 10\% for all populations tested. There also exist a significant relationship between fin flexibility and the generated thrust, which holds true for different kinematical parameters tested. The reason for this relationship is that different bending modes related to flexibility resulted in higher lateral movement of the fin's trailing edge which in turm produce higher lateral separation between the Reversed-Karman vortex sheet. Flexibility distribution also affects the resistance of the fin to bending produced by inertial and hydrodynamical forces exerted to the fin, which has been associated with propulsor reversal phenomena in-between strokes. Different bending mode is also responsible to the existence of camber along the fin, making the fin's profile similar to that of traditional propellers which resulted in increase of fluid dynamics efficiency. Another observation is that frequency plays a larger role to thrust generation compared to velocity as the increase of net-thrust average due to frequency change is close to 50\%, compared to increase of net-thrust by changing the velocity: 6\%. However, the effect of bending modes also plays a role in frequency variation, as the most optimum configuration does not yield optimum result given a specific frequency, particularly at lower frequency, due to different frequency produce different chord-wise profile as a response to the given frequency. 
\section{Future Works}
The future sequel to this work may include:
\begin{enumerate}
    \item Flow visualisation of the wake structure behind the fin.

    While there are a lot of previous researches that delves into the effect of fin flexibility to thrust generation visually, it will be interesting to observe the wake structure behind the fin to better understand the mechanism of thrust generation associated with flexibility.
    \item Checking the optimum kinematical parameters for different fin stiffness.

    Different fin stiffness produce different results which might indicates that for every stiffness configuration, there exist an optimum swimming condition for a given fin stiffness.
    \item Study the three dimensionality effect of thrust generation

    It is argued that it is not the stiffness of a fin that relates to thrust generation, instead it is the flexural flexibility (multiplication of second moment of inertia with Young's modulus) that is related to the thrust. It may be interesting to see the effect of moment of inertia and finding the optimum value of the moment of inertia, which indirectly relates to optimising the shape of the fin instead of using simple rectangle fin.
\end{enumerate}