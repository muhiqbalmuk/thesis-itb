\lstset{
    language=Matlab,%
    basicstyle=\ttfamily,
    breaklines=true,%
    keywordstyle=\color{blue},%
    identifierstyle=\color{black},%
    stringstyle=\color{yellow},
    commentstyle=\color{green},%
    showstringspaces=false,%without this there will be a symbol in the places where there is a space
    numbers=left,%
    numberstyle={\tiny \color{black}},% size of the numbers
    numbersep=9pt, % this defines how far the numbers are from the text
    emph=[1]{for,end,break},emphstyle=[1]\color{red}, %some words to emphasise
    %emph=[2]{word1,word2}, emphstyle=[2]{style},    
}
\begin{lstlisting}[caption={Pseudocode for main DAQ Code in MATLAB}]
clc
clear all
close all

%% INITIALISING
b = 0.0018;
m = 6.92;

ai = daq.createSession('ni');
addAnalogInputChannel(ai,'Dev2', 'ai0', 'Voltage');
ai.Rate = 500;

disp('Calculating zero load balance...');

[data,time] = ai.startForeground;

zerobal = mean(data);

disp('Begin the flapping motion in Arduino!')
pause(8);

%% DATA COLLECTING
disp('Collecting data...');

ai.DurationInSeconds = 13;

[volt,time] = ai.startForeground;

volt = (volt - zerobal)*1E3;
forceNormal = (volt - b)/m*9.81*1E-3;

%% FILTERING & SMOOTHING
% Filter design using 4th order Butterworth filter 
n = 4;
cutoffFreq = 2;
wn = 2*cutoffFreq/ai.Rate;
[z,p] = butter(n,wn);

% Actual filtering process
filterVolt = filter(z,p,volt);
filterforceNormal = filter(z,p,forceNormal);

% Smoothing process using moving average method up to 100 pts forward
smoothforceNormal = smooth(filterforceNormal,500,'moving');
forceAverage = mean(smoothforceNormal);

%% PLOTTING
figures = plot(time,smoothforceNormal);
xlabel('Time (s)');
ylabel('Normal Force (N)');
grid on;

%% SAVING
evalCase = 1;
save(['Experiment_Result',num2str(evalCase),'.mat'],'time','smoothforceNormal');
savefig(['Experiment_Result',num2str(evalCase),'.fig'],'figures');
\end{lstlisting}

\lstset{
    language=C++,%
    basicstyle=\ttfamily,
    breaklines=true,%
    keywordstyle=\color{blue}\ttfamily,
    stringstyle=\color{red}\ttfamily,
    commentstyle=\color{green}\ttfamily,
    morecomment=[l][\color{magenta}]{\#}
    showstringspaces=false,%without this there will be a symbol in the places where there is a space
    numbers=left,%
    numberstyle={\tiny \color{black}},% size of the numbers
    numbersep=9pt, % this defines how far the numbers are from the text
    emph=[1]{for,end,break},emphstyle=[1]\color{red}, %some words to emphasise
    %emph=[2]{word1,word2}, emphstyle=[2]{style},    
}
\begin{lstlisting}[caption={Code for traversing module in Arduino}]
  /* Sweep
  by BARRAGAN <http://barraganstudio.com>
  This example code is in the public domain.

  modified 8 Nov 2013
  by Scott Fitzgerald
  http://www.arduino.cc/en/Tutorial/Sweep

  Modified 7 Jul 2018
  by MI
  change delay to millis()
*/

#include <Servo.h>
#include <AccelStepper.h>

Servo myServo;
AccelStepper myStepper (1, 10, 11);

const int servoMinDegrees = 75;
const int servoMaxDegrees = 105;
const int servoCenterDegree = 90;
const int servoPin = 9;
const int stepperPulsePin = 10;
const int stepperDirPin = 11;

float servoPosition = 90;
/*
 * Set frequency in servoInterval
 * servoInterval = 8.325/desiredFrequency
 */
int servoInterval = 8;
float servoDegrees = 1;

unsigned long timeInit, timeFinal, timeElapsed;
unsigned long currentMillis = 0;
unsigned long previousServoMillis = 0;

void setup() {
  myServo.attach(servoPin);
  myServo.write(servoPosition);
  myStepper.setMaxSpeed(6000);

  Serial.begin(9600);// rate at which the arduino communicates
  Serial.println("TOWING TANK 2.0 MOTION MODULE");
}

void loop() {
  /*
   * Options: put 1, 2, or 3
     1. Flapping forward (moving away from user)
     2. Backward w/o flapping (moving toward user)
     3. Centering
  */
  int option = 1;
  if (option == 1) {
    currentMillis = millis();
    rotateStepperForward();
    rotateServo();
  }
  else if (option == 2) {
    rotateStepperBackward();
  }
  else if (option == 3) {
    myServo.write(servoCenterDegree);
  }
}

void rotateStepperForward() {
  /*
   * moveTo(8000) = 10 cm
   * setSpeed(800) = 1 cm/s
   */
  if (myStepper.distanceToGo() == 0) {
    myStepper.moveTo(8000);
    myStepper.setSpeed(800);
  }
  myStepper.runSpeedToPosition();
}

void rotateStepperBackward() {
  if (myStepper.distanceToGo() == 0) {
    myStepper.moveTo(-8000);
    myStepper.setSpeed(800);
  }
  myStepper.runSpeedToPosition();
}

void rotateServo() {
  /*
    This is similar to the servo sweep example except that it uses millis() rather than delay()
    nothing happens unless the interval has expired
    the value of currentMillis was set in loop()
  */
  if (currentMillis - previousServoMillis >= servoInterval) {
    // its time for another move
    previousServoMillis += servoInterval;

    servoPosition = servoPosition + servoDegrees; // servoDegrees might be negative

    if ((servoPosition >= servoMaxDegrees) || (servoPosition <= servoMinDegrees))  {
      // if the servo is at either extreme change the sign of the degrees to make it move the other way
      servoDegrees = - servoDegrees; // reverse direction
      // and update the position to ensure it is within range
      servoPosition = servoPosition + servoDegrees;
    }
    // make the servo move to the next position
    myServo.write(servoPosition);

    // and record the time when the move happened
    Serial.println(myServo.read());
  }
  else if (currentMillis > 30000) {
    myServo.write(servoCenterDegree);
  }
}
//=====END
\end{lstlisting}
