\begin{center}{
    \textbf{OPTIMISASI GAYA DORONG DARI SEBUAH PANEL EKOR DENGAN KEKAKUAN BERVARIASI SECARA EKSPERIMENTAL MENGGUNAKAN \textit{GLOBAL SURROGATE-ASSISTED GENETIC ALGORITHM} DAN \textit{KRIGING}}\\
    oleh\\
    \textbf{
        Muhammad Iqbal\\
        23616033\\
        (Program Studi Teknik Dirgantara)
    }
}
\end{center}
Tesis ini mendiskusikan tentang pengaruh kekakuan pada sebuah panel ekor dengan kekakuan yang variatif pada gaya dorong total yang dihasilkan oleh panel tersebut. Sebuah algoritma optimisasi, \textit{global surrogate-assisted genetic algorithm} dengan \textit{kriging} digunakan untuk mengoptimalkan gaya dorong total yang dihasilkan oleh model ekor, dimana \textit{kriging} digunakan untuk membentuk sebuah model pengganti sehingga proses optimisasi dapat dipercepat. Nilai \textit{fitness} dievaluasi dengan melakukan pengukuran gaya dorong secara langsung pada eksperimen \textit{towing tank}. Model panel ekor dibuat dari karet silikon yang didalamnya terdapat enam buah kawat pegas baja. Kekakuan dari model tersebut dapat divariasikan dengan mengubah panjang dari enam buah kawat tesebut, sehingga panel ekor tersebut memiliki kekakuan dari 5.5 MPa hingga 200 GPa. Data yang didapat menunjukkan bahwa ada sebuah kondisi kekakuan optimum yang menghasilkan gaya dorong total yang paling besar. Kondisi kekakuan ini tetap berlaku untuk kondisi parameter kinematika yang berbeda-beda. Namun, konfigurasi kekakuan panel ekor juga berubah apabila frekuensi dari gerakan mengepak ekor tersebut juga berubah, dengan frekuensi kepak yang rendah membutuhkan kekakuan yang rendah pula untuk menghasilkan gaya dorong total yang paling optimum. Disajukan pula bahwa ada sebuah parameter kekakuan yang lebih berpengaruh pada pembangkitan gaya dorong total yang optimal apabila dibandingkan dengan parameter kinematika lainnya.\par
\textit{Kata kunci: dinamika fluida eksperimental, algoritma genetik, \textit{kriging}, model pengganti, panel fleksibel}