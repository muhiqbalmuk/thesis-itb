James Gleick said that an experimenter has to muster graduate students, cajole machinists, and flatter lab assistants. I would like to dedicate this section to those who I flatter, cajole, and muster in order to complete my thesis.\par
My utmost gratitude obviously goes to The Creator and two of His: my parents. Their love and support for me has been without equal, and this thesis is, in a way, an expression of my gratitude and respect for them.\par
My academic life would be unnecessarily difficult if not for the assistance from my advisors and lecturers: Dr. Lavi R. Zuhal for his guidance on my thesis and life in general and Prof. Tatacipta Dirgantara for his insight on the structural matters on my thesis. Dr. Rianto A. Sasongko for trusting me to tutor students in Engineering and Advanced Mathematics. Also my thesis examiners Dr. Djoko Sardjadi and Dr. Firman Hartono for their questions and advice during my defense that makes this thesis so much better. I also cannot stress enough the help from my faculty's amazing machinists, technicians, and seniors in the creation of my thesis: Mas Adi, Pak Dali, Mas Mifta, Mas Eko, Pak Pepi, and Reyner for their experience and expertise in machining and electronics. Also Pak Imin, Teh Erlin, Pak Jupri, and Pak Yadi on their help in administrative matters.\par
I would also like to acknowledge Wisnu, Faber, Horas, Tope, Alif, Rilis, Putra, Luqman, Bintang, Aul, Mas Handoko, and Himpu who share my happiness and miseries, my long-distance best friends Dita, Sasha, and Jati for reminding me that thesis is not the only part of my life, and my dorm caretaker and her son, Bu Nia and Kiki for their help in my daily living.\par
\begin{minipage}{1.0\linewidth}
\begin{flushright}                                      
Bandung, 27 September 2018 \\ [2cm]
Muhammad Iqbal
\end{flushright} 
\end{minipage}