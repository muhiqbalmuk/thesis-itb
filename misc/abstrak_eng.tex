\begin{center}{
    \textbf{EXPERIMENTAL THRUST OPTIMISATION OF A VARIABLE-STIFFNESS FIN PANEL USING GLOBAL SURROGATE-ASSISTED GENETIC ALGORITHM WITH KRIGING}\\
    by\\
    \textbf{
        Muhammad Iqbal\\
        23616033\\
        (Aerospace Engineering Programme)
    }
}
\end{center}
This thesis discuss how the flexibility of a variable-stiffness fin panel model affects the generated net thrust. An optimisation algorithm, global surrogate-assisted genetic algorithm with Kriging was used to optimise the net thrust force produced by the fin panel model, where kriging was used to construct a surrogate model to accelerate the optimisation process. Fitness value was evaluated by directly measuring the thrust force in towing tank experiment. Fin panel model was created using silicon rubber embedded with six spring wires. The fin panel's flexibility can be varied by changing the length of six springs embedded within the silicon rubber, which enables the fin panel to have flexibility range of 0.08 GPa to 207 GPa. The data shows that there exist an optimum stiffness condition which holds true for different kinematical parameters variation. However, different stiffness configurations also change the optimum frequency of the fin, with lower frequency fin responds favorably to lower stiffness configuration fin due to different bending response for a given material properties. It is also shown that there exist a more prominent kinematical parameter which produce the larger optimum thrust compared to the other, less sensitive, kinematical parameter.\par
\textit{Keywords: experimental fluid dynamics, genetic algorithm, kriging, surrogate modeling, flapping flexible panel}
